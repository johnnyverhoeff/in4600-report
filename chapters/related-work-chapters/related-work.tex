% !TeX root = ../../thesis.tex

\chapter{Related Work \& Proposed Method}
\label{chp:related-work}



% normal energy dissasgregation work with sigatures.
% normal meters not able to distinguis lightinh .... ?? 
% meters only used for lighting can say X amount for lighting, but noi finer granularityl
% some researh able to identify a sigle light, bit no more due to too similar LED signatures.

% other plausbule soluations -> PoE (dont critisze), PLC

% my approach still dissagregtraion instead of node on network, and give the unique signatures with VLC













This chapter first describes the state-of-the-art in energy disaggregation research and their results with the disaggregation of individual lights.
Next, other methods are discussed, which could hypothetically identify if a light is on or off.
And finally the proposed method is introduced.


	\section{Related Work}

		Energy disaggregation can categorize devices such as a refrigerator or washer dryer, but it cannot disaggregate two devices which have the same signature.
		For example: The same lights in different rooms could not be disaggregated in the work performed by the authors of \cite{froehlich2011disaggregated}.
		In other words, if there is more than one light in a house, and a subset of those lights are on, it cannot be identified which subset of lights are actually on, only that there are some lights on.


		The authors of \cite{shao2013temporal} have shown that their method is able to disaggregate up to three separate lights.
		The key difference here is that the lights have different power ratings.
		Because the lights were different and had different power ratings, these lights could be disaggregated from each other.
		Nothing is mentioned about how the disaggregation results would change, if there were multiple devices used with the same power ratings, for example three identical lights.


		In \cite{Gupta:2010} the authors tried to disaggregate appliances which were powered with a switching mode power supply (SMPS).
		These power supplies continuously generate high frequency electromagnetic interference (EMI).
		Appliances can be distinguished based on the differences between the switching frequencies characteristics of each SMPS.
		The method of disaggregating appliances which are powered with a SMPS requires training data of the characteristics of each SMPS.  
		Many lights also use some kind of SMPS, due to the high efficiency of these types of power supplies.
		It is claimed that two appliances of the same brand and model can be distinguished by looking at the EMI from the SMPS.
		However it was also observed that the system could not distinguish similar light fixtures when they were installed spatially near each other.
		The EMI from the power supplies did not have sufficient differences to correctly distinguish between these similar lights. 


		In \cite{Englert:2016:LAE:2893711.2893724} the authors present a method which correlates the light switching events obtained from a Building Control Network (BCN) with the power measurements on the mains distribution level.
		A BCN can manage the lights and HVAC in a building.
		It can turn the lights on or off depending on the time of day, for example.
		So this work already knows which lights are on and off. 
		This work uses the combined sources of the BCN and the power measurements to estimate the electricity consumption for each light individually, with an error below 11 \%.
		They use the BCN to retrieve which lights are being switched on or off and then they look at how the power consumption of the building changes.
		In this way they can estimate what the power consumption of each individual light is, without having explicit training or manual inputs.
		The system adapts itself to the characteristics of any building within 8 to 14 days.



		

	\section{Hypothetical Methods}

		In this section other methods will be discussed that can potentially identify which lights are on and off, by other means than looking at the energy signature.




		\subsection{Power-line Communication}

		Power-line communication or PLC is a communication method that uses the existing AC wiring to simultaneously carry both data and AC power \cite{1205458}.
		The obvious benefit is that the existing wiring infrastructure can be used.
		Data is transmitted with a frequency that is much higher than the 50 or 60 Hz AC power frequency in order to ensure that the power wave does not interfere with the data signal.
		Each light should be equipped with a modulator that can transmit the information about a light over the powerline.
		Then a receiver, can decode the information that is transmitted by the modulators.
		Multiple lights can then transmit their status via for example orthogonal frequency-division multiplexing (OFDM) \cite{hoch2011comparison}.




		\subsection{Power over Ethernet}

		Power over Ethernet or PoE is a standard which passes DC power along with data on an Ethernet cable \cite{patoka2003power}.
		With this solution each light becomes a node in a network with an Ethernet cable being connected to the light.
		The PoE technique provides the power to use the light and the Ethernet standards allow a micro-controller to transmit status information about that light to a central server, for example.
		The number of lights that can transmit their status is limited by the network protocols used.
		A drawback is that an Ethernet cable must be used for each light instead of the existing AC wiring.
		Another drawback is that PoE provides low voltage DC power and so the power dissipated in the cable itself will also be greater than with the existing high voltage AC wiring.




	\section{Proposed Method}

	\begin{figure}[h]
		\centering
		\resizebox {\textwidth} {!} {
			\begin{tikzpicture}

				\node[block] (smart_meter) {Smart-meter};	

				\node[block, left = 1cm of smart_meter] (power) {Power};	
				\draw[line] (power.east) -- (smart_meter.west) node [midway, right] {};		

				% second mod.
				\node[block, right = 4cm of smart_meter] (second_modulator) {Modulator};
				\node[LED, right = 0.5cm of second_modulator] (second_led) {LED};
				\draw[line] (second_modulator.east) -- (second_led.west) node [midway, right] {};

				% first mod.
				\node[block, above = 1cm of second_modulator] (first_modulator) {Modulator};
				\node[LED, right = 0.5cm of first_modulator] (first_led) {LED};
				\draw[line] (first_modulator.east) -- (first_led.west) node [midway, right] {};

				% third mod.
				\node[block, below = 1cm of second_modulator] (third_modulator) {Modulator};
				\node[LED, right = 0.5cm of third_modulator] (third_led) {LED};
				\draw[line] (third_modulator.east) -- (third_led.west) node [midway, right] {};

				\node[dot, right = 2.5cm of smart_meter] (CP) {};

				\draw (smart_meter.east) -- (CP) node [pos=0.3, above] {0222};

				\draw (CP) |- (first_modulator.west) node  [pos=0.75, above] {0011};
				\draw (CP) |- (second_modulator.west) node [pos=0.75, above] {0110};
				\draw (CP) |- (third_modulator.west) node  [pos=0.75, below] {0101};

			\end{tikzpicture}
		}
		\caption{Block diagram how each component interacts with other components.}
		\label{fig:overview-diagram}
	\end{figure}



	If a building uses PoE, the monitoring of lights is trivial.
	But many legacy buildings do not have this infrastructure.
	The proposed method aims to still work in legacy buildings, so it must use the legacy AC wiring that is already present.
	Since the method is aimed at legacy buildings, we cannot rely on a Building Control Network (BCN).
	A single smart-meter is used to measure the energy usage.
	The purpose of the smart-meter is to identify which lights are on and which are off.
	It does this by looking at the the energy signature of each light.
	But as discussed many lights have the same signature, so it is not possible to accurately identify which lights are on using only a smart-meter.

	So besides a smart-meter, smart-lights will also be used.
	These smart-lights can take advantage of Visible Light Communication (VLC).
	VLC is a short-range optical means of wireless communication using the visible light spectrum.
	VLC is made possible by the advances in LED technology.
	VLC can turn the lights on or off at high frequencies to avoid seeing flickering effects.
	The interest in VLC has grown since the widespread deployment of LED lighting fixtures for energy efficiency over the normal incandescent light bulb \cite{rajagopal2012ieee}.


	Visible light communication opens up the opportunity to monitor the energy consumption of lights. Applications such as indoor localization are becoming widely popular with VLC and these applications require lights to broadcast a unique ID. 
	Given that VLC modulates LEDs by turning them on and off, the ID that is used for indoor localization can also be used for detecting if the light is on or off by looking at the current signature.

	%Now that VLC allows lighting fixtures to transmit data, research is being done to make indoor localization more precise than with traditional RF-based approaches \cite{Kuo:2014:LIP:2639108.2639109}.
	%Since the light does not pass through walls and it has less reflection compared to RF, an indoor localization system can be more precise.
	%Each VLC lighting fixture can act like a beacon which transmits an ID of some sort, thereby describing its location inside the building.

	%So the lights are already transmitting IDs to describe their location inside a building with the help of VLC.
	The information that a light is transmitting via VLC, will also propagate to the current that the light draws.
	And with the smart-meter, we are looking at the aggregated current draw of all attached devices, including these lights.
	The information or ID that the light will be transmitting, needs to be constructed in such a way that the sum of all these currents, which flows through the smart-meter, can still tell us which lights are actually on or off.
	How these IDs can be constructed needs be to be investigated.
	With this proposed method the lights can still be used as beacons for applications such as indoor localization because the IDs will still be unique.
	The current signature of each light will tell us which lights are on and it can be done over the existing wiring infrastructure.

	In \autoref{fig:overview-diagram} an overview of the method can be seen.
	The smart-meter is shown along with multiple smart lights. 
	Each smart light is a standalone device which has the necessary embedded hardware to modulate the light.
	The IDs that will be used can also be seen next to the smart lights.
	The sum of these IDs will then flow as current through the smart-meter, which will sample this aggregated current and in turn be able to identify which of the LEDs is on and which is off.


