% !TeX root = ../../thesis.tex

\chapter{Related Work \& Proposed Method}
\label{chp:related-work}



% normal energy dissasgregation work with sigatures.
% normal meters not able to distinguis lightinh .... ?? 
% meters only used for lighting can say X amount for lighting, but noi finer granularityl
% some researh able to identify a sigle light, bit no more due to too similar LED signatures.

% other plausbule soluations -> PoE (dont critisze), PLC

% my approach still dissagregtraion instead of node on network, and give the unique signatures with VLC













This chapter first describes state-of-the-art in energy disaggregation research and their results with the disaggregation of individual lights.
Next, other methods are discussed, which could hypothetically identify for each light in the system, if it is on or off.
And finally the proposed method is introduced.


	\section{Related Work}

		Energy disaggregation can categorize devices such as a refrigerator or washer dryer, but it cannot disaggregate two devices which have the same signature, for example the same lights in different rooms \cite{froehlich2011disaggregated}.
		In other words, if there is more than one light in a house, and a subset of those lights are one, it cannot be identified which subset of lights are actually on, only that there are some lights on.


		\cite{shao2013temporal} has shown that is is able to disaggregate up to three separate lights.
		The key difference here is that the lights have different power ratings.
		Because the lights were different and had different power ratings, these light could be disaggregated from each other.
		Nothing is mentioned about how the disaggregation results would alter, if there were multiple devices used with the same power ratings, for example three identical lights.


		\cite{Gupta:2010} tried to disaggregate appliances which were powered with a switching mode power supply (SMPS).
		These power supplies continuously generate high frequency electromagnetic interference (EMI).
		Appliances can be distinguished based on the differences between the switching frequencies characteristics of each SMPS.
		Many lights also use some kind of SMPS, due to the high efficiency of these types of power supplies.
		It is claimed that two appliances of the same brand and model can be distinguished by looking at the EMI from the SMPS.
		However it was also observed that the system could not distinguish similar light fixtures when they were installed spatially near each other.
		The EMI from the power supplies did not have sufficient differences to correctly distinguish between these similar lights. 


		

	\section{Hypothetical Methods}

		In this section other methods will be discussed that can potentially identify which lights are on and off, by other means than looking at the energy signature.





		\subsection{Power-line Communication}

		Power-line communication or PLC is a communication method that uses the existing AC wiring to simultaneously carry both data and AC power \cite{1205458}.
		The obvious benefit is that the existing wiring infrastructure can be used.
		Data is transmitted wit ha frequency that is much higher than the 50 or 60 Hz AC power frequency in order to ensure that the power wave does not interfere with the data signal.
		Multiple lights can then transmit their status via for example orthogonal frequency-division multiplexing (OFDM) \cite{hoch2011comparison}.




		\subsection{Power over Ethernet}

		Power over Ethernet or PoE is a standard which passes DC power along with data on an Ethernet cable \cite{patoka2003power}.
		With this solution each light becomes a node in a network with an Ethernet cable being connected to the light.
		The PoE technique provides the power to use the light and the Ethernet standards allows a micro-controller to transmit status information about that light to a central server, for example.
		The number of lights that can transmit their status is limited by the network protocols used.
		A drawback is that an Ethernet cable must be used for each light instead of the existing AC wiring.
		Another drawback is that PoE provides low voltage DC power and so the power dissipated in the cable itself will also be greater than with the existing high voltage AC wiring.




	\section{Proposed Method}


	The idea for this thesis was to piggyback on Visible Light Communication (VLC).
	VLC is a short-range optical means of wireless communication using the visible light spectrum.
	VLC is made possible by the advances in LED technology.
	The interest in VLC has grown since the widespread deployment of LED lighting fixtures for energy efficiency over the normal incandescent light bulb \cite{rajagopal2012ieee}.


	Now that VLC allows lighting fixtures to transmit data, research is being done to make indoor localization more precise than with traditional RF-based approaches \cite{Kuo:2014:LIP:2639108.2639109}.
	Since the light does not pass through walls and it has less reflection compared to RF, an indoor localization system can be more precise.
	Each VLC lighting fixture can act like a beacon which transmits an ID of some sort, thereby describing its location inside the building.


	Many research papers have successfully recognized home appliances through energy disaggregation with data being collected with a single smart-meter.   
	Though they succeed in detecting appliances with unique signatures, lighting is put in a group without mentioning the individual lights \cite{kolter2011redd}.
	The reason for this, is because every light has a very similar power draw and therefor a very similar signature and it can not be reliably disaggregated.


	So if the lights are already transmitting IDs to describe their location inside a building with the help of VLC, we can piggyback on VLC by looking at the current signatures that these lights produce.
	The IDs need to be constructed in such a way that the sum of all these currents can still tell us which of the lights are actually on or off.
	With this proposed method the lights can still act as beacons for indoor localization because the IDs will still be unique and the current signature of each light will tell us which lights are on and it can be done over the existing DC or AC wiring infrastructure.


