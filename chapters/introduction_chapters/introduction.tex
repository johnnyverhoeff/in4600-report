% !TeX root = ../../thesis.tex

\chapter{Introduction}
\label{chp:introduction}

\vspace{1\baselineskip}

\noindent

%\vspace{1\baselineskip}

In a world where the vast majority of energy consumed, is provided by unsustainable fossil fuels~\cite{kolter2011redd}, it is needed to know exactly what that energy is being used for. 
If we know what that energy is being used for, we can determine if we really need it for that device at that particular time.
If that device is not needed at that time, the energy used to power it, is essentially being wasted.




Normal meters give an aggregate power consumption, which cannot tell a consumer which specific device is responsible for its energy consumption or waste thereof.
This is where energy disaggregation comes in.
Energy disaggregation aims to break down the aggregate power consumption of, for example a household, to tell the consumer which devices consumes power at which times.




The method of energy disaggregation has been applied successfully for energy consumers such as refrigerators and HVAC~\cite{kolter2011redd} \cite{spiegel2014energy}.
But other energy consumers such as lighting has not been so successful~\cite{spiegel2014energy} \cite{batra2015neighbourhood}.
The reason for this, is that most lighting has no unique signature, which a microwave or a refrigerator do have.
It is important to be able to disaggregate the energy consumed by lighting, because it is the third largest energy consumer in the average home~\cite{batra2015neighbourhood}.




If every light in a household has a unique signature, that is distinguishable from all other lights, the consumer can get better insight in how the lighting is being used in his household. 
With that information, lights that are being used to light a room in the middle of the day for example, can be turned off.
Thereby saving energy and costs.



% !TeX root = ../../thesis.tex


\section{Problem Definition}

Problem Definition \\
Problem Definition \\
Problem Definition \\

% !TeX root = ../../thesis.tex


\section{Thesis contributions}

Thesis contributions \\
Thesis contributions \\
Thesis contributions \\

% maybe more ..

% !TeX root = ../../thesis.tex

\section{Thesis organization}

Thesis Organization \\
Thesis Organization \\
Thesis Organization \\









