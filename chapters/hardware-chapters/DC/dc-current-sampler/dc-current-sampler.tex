% !TeX root = ../../../../thesis.tex

 \subsection{Current Sampler}

	Now that the hardware is created to translate the IDs of the LEDs into a unique current signature, we also need a way to measure the current.
	The measured current can then be processed by a micro-controller.
	And in turn it can be identified which LEDs are on and which are off.

	In the interest of time the most simple manner was chosen to measure the current for the DC hardware.
	Other options are available for measuring current and they will be discussed in the AC part.
	The most simple way to measure current is by using a series resistor.
	The resistance does not variate and therefor no noise is introduced in the sampled signal.
	The resistor is placed in series between the DC power supply and the LEDs.
	The voltage drop over the resistor is linearly proportional to the current that flows through the resistor, according to Ohm's Law $U = R \times I$.
	If the value of the resistor is chosen such that the maximum voltage will never exceed the rated voltage for a micro-controller, it can be directly measured by the micro-controller's ADC in question.