% !TeX root = ../thesis.tex

\chapter{Conclusions and Future Work}
\label{chp:conclusionsandfuturework}

\section{Conclusions}


The aim of this thesis is to find out if lights could be identified as being on or off, through the current signature, with the help of VLC and a single smart-meter.
CDMA codes have been investigated and compared to see which is the best suited for this scenario.
Two solutions have been discussed to overcome the interference problem that comes with these types of codes.
When these codes were understood and made usable in software simulations, two practical testbed were developed, for DC and AC, in order to experiment with.
When the correct codes are used dependent on the size of the system, each individual light in each testbed could be successfully identified as being on or off in a timely manner.
For larger systems a simulation was performed which shows that there can be made a trade-off between time and accuracy. 
But the simulation showed that even with a high accuracy, the lights could still be identified in a timely manner.
%As the testbed only represents a case where only these lights were connected, there is more work to be done when for instance there are also other appliances connected.






\section{Future Work}


This work is only the first step to disaggregate which lights are on and off.
There remains more work to be done, below there are ideas for future work.


\subsection{Other Appliances}

As the testbeds represents a case where only these lights were connected, the question rises: What will happen when other appliances are connected ?
These could for example be an incandescent light bulb or a refrigerator.
To answer that question sample data from \cite{kolter2011redd} could be taken to represent some household appliances and added with the data of the modulation shown from the testbeds.
Then signal filtering techniques could be performed to try and filter the signal from the modulating lights, given the fact that we know that these lights operate at a certain constant frequency.


\subsection{Dimming Lights}

LEDs can be often too bright for a persons liking, so the lights are dimmed.
Dimming of an LED can be done in two ways: PWM, by lowering the duty cycle and so less power is dissipated by the LED or by limiting the current that flows through the LED.
Since the LEDs are modulating via an OOK scheme, PWM cannot be used so instead current limiting must be done in order to dim the lights.
But the CDMA codes are designed to work with each transmitter or LED has the same amplitude or in this case the same current.
By dimming the LED and changing the current the codes may not work anymore.
A solution for this can be that each LED will have multiple dimming levels and that for each dimming level other frequencies are used to modulate at.
A filter can then be used to filter between these LEDs which all have the same amplitude.





%power limiting capacitor...




