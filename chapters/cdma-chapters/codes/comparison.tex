% !TeX root = ../../../thesis.tex

\subsection{Comparison of Sequences}
\label{subsec:comparison-of-sequences}


In \autoref{tbl:comparison-sequences} a comparison is made between the CDMA sequences discussed.
The sequences are compared for the metrics discussed in \autoref{sec:performance-metrics-cdma} in both synchronous as well as asynchronous environments.
From this table it is clear that the Gold sequences are the best sequences suited for the environment of disaggregating individual lights, due to their ability to work when transmitting asynchronously, low cross-correlation, one-peaked auto-correlation and their scalability.







\begin{table}[h!]
	\centering
	\begin{tabular}{  | l | l | l | l | }

		\hline
														& Orthogonal			& PN 						& Gold				\\ \hline
		Synchronous	Transmission						& \cmark				& \cmark					& \cmark				\\ \hline
		Asynchronous Transmission						& \xmark				& \cmark					& \cmark				\\ \hline
		%Peaks auto-correlation (synchronous)			& 1						& 1							& 1						\\ \hline
		%Peaks auto-correlation (asynchronous)			& $> 1$					& 1							& 1						\\ \hline
		%Low cross-correlation (synchronous)			& \cmark				& \cmark					& \cmark				\\ \hline
		%Low cross-correlation (asynchronous)			& \xmark				& \cmark					& \cmark				\\ \hline
		Math. bounded cross-corr. (synchronous)			& \cmark				& \xmark					& \cmark				\\ \hline
		Math. bounded cross-corr. (asynchronous)		& \xmark				& \xmark					& \cmark				\\ \hline
		Scalability ($C \propto L$)						& \cmark				& \xmark					& \cmark				\\ \hline				



	\end{tabular}
	\caption{Table showing a comparison of the discussed CDMA sequences. }
	\label{tbl:comparison-sequences}

\end{table}