% !TeX root = ../../../thesis.tex

\subsection{Comparison of Sequences}
\label{subsec:comparison-of-sequences}


In \autoref{tbl:comparison-sequences} a comparison is made between the CDMA sequences discussed.
The sequences are compared for the metrics discussed in \autoref{sec:performance-metrics-cdma} in both synchronous as well as a-synchronous environments.
From this table it is clear that the Gold sequences are the best sequences suited for the environment of disaggregating individual lights, due to their a-synchronous nature, low cross-correlation and one-peaked auto-correlation and their scalability.







\begin{table}[h!]
	\centering
	\begin{tabular}{  | l | l | l | l | }

		\hline
														& Orthogonal Seq. 			& PN Seq.						& Gold Seq.				\\ \hline
		Synchronous	Transmission						& \cmark					& \cmark						& \cmark				\\ \hline
		A-synchronous Transmission						& \xmark					& \cmark						& \cmark				\\ \hline
		Peaks auto-correlation (synchronous)			& 1							& 1								& 1						\\ \hline
		Peaks auto-correlation (a-synchronous)			& $> 1$						& 1								& 1						\\ \hline
		Low cross-correlation (synchronous)				& \cmark					& \cmark						& \cmark				\\ \hline
		Low cross-correlation (a-synchronous)			& \xmark					& \cmark						& \cmark				\\ \hline
		Math. bounded cross-correlation (synchronous)	& \cmark					& \xmark						& \cmark				\\ \hline
		Math. bounded cross-correlation (a-synchronous)	& \xmark					& \xmark						& \cmark				\\ \hline
		Scalability ($C \propto L$)						& \cmark					& \xmark						& \cmark				\\ \hline				



	\end{tabular}
	\caption{Table showing a comparison of the discussed CDMA sequences. }
	\label{tbl:comparison-sequences}

\end{table}