% !TeX root = ../../thesis.tex

\section{Performance Metrics of a CDMA Sequence}
\label{sec:performance-metrics-cdma}

To be able to objectively determine which code sequence is the best for certain environments, metrics are needed to compare the performance of a sequence.


Such metrics are: 

\begin{itemize}

	\item Auto- and cross-correlation: This will determine how accurate we can detect the ID of an LED in the aggregated signal, and also the amount of interference there will be from the other IDs.

	\item Code length: If the code is too long, the system will not be able to detect an LED in a timely manner.

	\item Number of codes: The number of unique codes that can be used in the same system is important for the scalability.

	\item Asynchronous behavior: It should also be considered if the codes can be used in a synchronous only or in an a-synchronous environment.

\end{itemize}



Correlation is a measure for determining how much sequence $X$ is similar to sequence $Y$ and can be found in \autoref{eq:correlation}.
With $L$ being the length of the code and $\tau$ the time-shift.
When sequence $X$ and $Y$ are the same sequence, we speak of the auto-correlation.
When they are two different sequences, we speak of the cross-correlation. 

\begin{equation}
	R(\tau)_{xy} = \displaystyle\sum_{i = 0} ^ {L - 1} x(i) \times y(i + \tau) {\text{  with $\tau = 0, 1, 2, \dotsc, L$}}
	\label{eq:correlation}
\end{equation}


%commented out because this will never be used and therefor only adds unnecessary text

%Another way to calculate the correlation between two sequences is to count the number of agreements and disagreements between the two sequences, see \autoref{eq:correlation-a-d}, this comes in handy when comparing two digital sequences, which both have $0$ and $1$ signal levels.

%\begin{equation}
%	R(\tau) = \text{\# of agreements} - \text{\# of disagreements} 
%	\label{eq:correlation-a-d}
%\end{equation}




The properties of an ideal set of codes should be, that the auto-correlation for each code should have a clear peak to identify that this code is present in the signal.
This peak should only occur when there is no time-shift, so at $\tau = 0$.
The value of this peak should then be $L$, meaning that each chip of the code is equal to the value in the received signal.
When there is a time-shift, $\tau \neq 0$, the auto-correlation should be as close to zero as possible.
If the signal is the sampled current and the code is the ID of an LED, then we can say the LED is on when the auto-correlation peak is seen.
The ideal cross-correlation properties should be $0$ for every time-shift $\tau$, so that no code interferes with any other code, hereby causing no MAI (Multiple Access Interference).



The length of a code is also of importance, because each chip of the code has to be transmitted.
Assuming a constant modulating frequency, the time it takes to transmit a code sequence is proportional to the length of that code sequence.
The length of the code will also determine, to some extent, the number of codes that can be used together in the system.
These codes are said to be in the same set.
The number of codes in the same set determines the scalability of the system.




