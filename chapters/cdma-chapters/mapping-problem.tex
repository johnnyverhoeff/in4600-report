% !TeX root = ../../thesis.tex

\section{Mapping Problem}
\label{sec:mapping-problem}

The coding methods as discussed in \autoref{sec:orthogonal-sequences} and \autoref{sec:pn-sequences} are used in the field of telecommunication.
Since these signals are analog radio waves, the symbols are $+1$ and $-1$.
The orthogonal sequences generated via the Hadamard matrix can already be in this form, but the PN sequences contain only zeros and ones.
For the use with radio-signals the PN sequence is mapped to a form with $+1$ and $-1$, where the original $0$ is mapped to $+1$ and the original $1$ is mapped to $-1$.




The following proof \todo{Are we calling this a proof ??} shows how to calculate the correlation when the coding sequence consists of $+1$ and $-1$ symbols: 

\begin{proof}
	Let $s(t)$ be the received signal which is the composed signal of $m$ distinct codes.\\
	And let $c_i(t)$ be the code sequence $i$, and let $c_1(t)$ be the code for which we want to check if there is information there. \\

	\begin{align*}
		R(\tau)_{xy} = \displaystyle\sum_{t = 0} ^ {L - 1} x(t) \times y(t + \tau)	\tag{See \autoref{eq:correlation}}
		\\ \tau = 0,\ x = s(t),\ y = c_1(t)	
		\\ R(0)_{sc_{1}} = \displaystyle\sum_{t = 0} ^ {L - 1} s(t) \times c_1(t)
		\\ s(t) = \displaystyle\sum_{i = 1} ^ {m} c_i(t)														
		\\ R(0)_{sc_{1}} = \displaystyle\sum_{t = 0} ^ {L - 1}  \Bigg\{  c_1(t)	\times \displaystyle\sum_{i = 1} ^ {m} c_i(t) \Bigg\}
		\\ R(0)_{sc_{1}} = \displaystyle\sum_{t = 0} ^ {L - 1} \Bigg\{ c_1(t) \times c_1(t) + c_1(t) \times  \displaystyle\sum_{i = 2} ^ {m} c_i(t) \Bigg\} 
		\\ R(0)_{sc_{1}} = \displaystyle\sum_{t = 0} ^ {L - 1} c_1(t) \times c_1(t) + \displaystyle\sum_{t = 0} ^ {L - 1} \Bigg\{ c_1(t) \times  \displaystyle\sum_{i = 2} ^ {m} c_i(t) \Bigg\} 
		\\ R(0)_{sc_{1}} = L + \displaystyle\sum_{t = 0} ^ {L - 1} \Bigg\{ c_1(t) \times  \displaystyle\sum_{i = 2} ^ {m} c_i(t) \Bigg\} 
	\end{align*}

\end{proof}

So the result is $L$ plus the sum of all the other correlations. 
If the orthogonal sequences were used all the other correlations are zero.
If the Gold sequences were used each of the other correlations could be one of the values stated in \autoref{eq:corsscorr-gold}.
This is where the multiple access interference shows.



The above correlation calculation only holds up, when using a coding sequence which has the $+1$ and $-1$ symbols.
But when having LEDs which are modulating according to a sequence, the current draw is either on or off, a zero or a one.
This means a different correlation calculation is required.
First a formula for the mapping from $+1$ to zero and $-1$ to one is needed.
The formula can be found in \autoref{eq:radio-to-bin}, where $r$ denotes the $+1$ or $-1$ symbols and the outcome $b$ will be our binary value, i.e. $0$ or $1$.

\begin{equation}
	b = \frac{1 - r}{2}
	\label{eq:radio-to-bin}
\end{equation}


The previous proof can now be altered to incorporate the fact that the LEDs work with a one and zero state.


\begin{proof}
	Let $s(t)$ be the received signal which is the composed signal of $m$ distinct codes, which have symbols consisting of zero and one, which will be denoted be $c^b_i(t)$\\
	And let $c^r_i(t)$ be the code sequence $i$, and let $c^r_1(t)$ be the code for which we want to check if there is information there. \\
	Where $c^r_i(t)$ denotes a code sequence consisting of symbols $-1$ and $+1$.

	\begin{align*}
	T(\tau)_{xy} = \displaystyle\sum_{t = 0} ^ {L - 1} x(t) \times y(t + \tau)	\tag{See \autoref{eq:correlation}}
	\\ \tau = 0,\ x = s(t),\ y = c^r_1(t)
	\\ T(0)_{sc^r_{1}} = \displaystyle\sum_{t = 0} ^ {L - 1} s(t) \times c^r_1(t)	
	\\ s(t) = \displaystyle\sum_{i = 1} ^ {m} c^b_i(t)
	\\ T(0)_{sc^r_{1}} = \displaystyle\sum_{t = 0} ^ {L - 1} \Bigg\{  c^r_1(t)	\times \displaystyle\sum_{i = 1} ^ {m} c^b_i(t) 	\Bigg\}
	\\ T(0)_{sc^r_{1}} = \displaystyle\sum_{t = 0} ^ {L - 1} \Bigg\{  c^r_1(t)	\times \displaystyle\sum_{i = 1} ^ {m} \frac{1 - c^r_i(t)}{2}  	\Bigg\}
	\\ T(0)_{sc^r_{1}} = \frac{m}{2} \times \displaystyle\sum_{t = 0} ^ {L - 1} c^r_1(t) - \frac{1}{2} \times \displaystyle\sum_{t = 0} ^ {L - 1}  \Bigg\{ c^r_1(t) \times \displaystyle\sum_{i = 1} ^ {m} c^r_i(t) \Bigg\}
	\end{align*}

\end{proof}

This correlation calculation, denoted by $T$ now shows a different result than the previous result $R$.
Where $R = \displaystyle\sum_{t = 0} ^ {L - 1}  \Bigg\{  c_1(t)	\times \displaystyle\sum_{i = 1} ^ {m} c_i(t) \Bigg\}$ and $T = \frac{m}{2} \times \displaystyle\sum_{t = 0} ^ {L - 1} c^r_1(t) - \frac{1}{2} \times \displaystyle\sum_{t = 0} ^ {L - 1}  \Bigg\{ c^r_1(t) \times \displaystyle\sum_{i = 1} ^ {m} c^r_i(t) \Bigg\}$.
We can write it such that $R$ will become a function of $T$: 

\begin{proof}

	\begin{align*}
	T = \frac{m}{2} \times \displaystyle\sum_{t = 0} ^ {L - 1} c^r_1(t) - \frac{1}{2} \times R
	\\ R = m \times \displaystyle\sum_{t = 0} ^ {L - 1} c^r_1(t) - 2 \times T
	\end{align*}

\end{proof}

The sum of the sequence depends on the sequence used, when using orthogonal codes the sum will be zero, see \autoref{sec:orthogonal-sequences}.
If the sum is equal to zero, the factor $m$ will not matter.
However when the sequence used is a PN sequence or a balanced Gold sequence, the sum will be $-1$, This means: $R = -m - 2 \times T$, and then $m$, the number of signals, is important.
When using an unbalanced Gold sequence, the sum if that sequence can be calculated on the spot.









