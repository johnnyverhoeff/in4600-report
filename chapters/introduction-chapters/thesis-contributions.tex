% !TeX root = ../../thesis.tex


\section{Thesis Contributions}

The aim of this thesis is to propose a framework consisting of theoretical methods, hardware and software, so that lights with similar power draw can be distinguished with a single smart-meter.
This thesis takes advantage of advances in two areas: CDMA codes and VLC.

The specific contributions of this thesis are:

\begin{itemize}

	\item Coding methods are analyzed to allow each LED to have a unique power draw. 
	These codes make it possible to identify if an LED is on or off, even when multiple LEDs are modulating and thereby interfering with each others unique current signature.




	\item Hardware is introduced to allow the LEDs to be modulated by a micro-controller. 
	This will allow the LEDs to propagate their unique signatures via either AC or DC. 
	And there is also hardware introduced that can sample the current at a high frequency via either AC or DC.
	The sampled current is then processed by another separate micro controller, which can then state which LEDs are on and which or off.




	\item An evaluation of the proposed hardware and software is carried out, with a testbed which uses standard LED light fixtures. For larger scale evaluations, software simulation is used. 
\end{itemize}
