% !TeX root = ../../thesis.tex

\chapter{Introduction}
\label{chp:introduction}

\vspace{1\baselineskip}

\noindent

%\vspace{1\baselineskip}

In a world where the vast majority of consumed energy, is provided by unsustainable fossil fuels~\cite{kolter2011redd}, the amount of energy we use must be reduced.
Reducing energy consumption requires knowing which devices are actually consuming the energy.
When consumers are given feedback about their energy consumption, energy savings can be made, up to 15 \% \cite{darby2006effectiveness}.
If a consumer knows which devices are actually using energy, it can be decided if that devices really has to be used at that particular time.
If that device is not needed at that time, the energy used to power it, is essentially being wasted.
For example, lights on in a room which is not occupied.



Normal meters give an aggregate power consumption, which cannot tell a consumer which specific device is responsible for its energy consumption or waste thereof.
This is where energy disaggregation comes in.
Energy disaggregation aims to break down the aggregate power consumption of, for example a household, to tell the consumer which devices consumes power at which times.




Energy disaggregation has been applied successfully to identify the operation of appliances, such as refrigerators and HVAC \cite{kolter2011redd} \cite{spiegel2014energy}.
Identifying the energy consumption of other devices such as lighting has not been so successful \cite{spiegel2014energy} \cite{batra2015neighbourhood}.
Each appliance in a household draws power in a certain way, this can be thought of as a signature of that appliance.
As can be seen from \autoref{fig:energy-consumption-house}, appliances such as the washer dryer and the dishwasher can be distinguished from the aggregated power draw.
These appliances can be recognized by their signature: the amount of time they draw power, how large the power draw is and if it has a recurring pattern, for example in the case of the refrigerator in \autoref{fig:energy-consumption-house}.
But the lighting can not be disaggregated on a per lamp basis.
The reason for this, is that most lighting fixtures do not have a unique signature, instead many lights have the same signature.
Disaggregating the energy consumed by lighting is important, because it is the third largest energy consumer in the average home \cite{batra2015neighbourhood}.
In buildings, lighting consumes around 30 \% of the total energy consumption \cite{halonen2010guidebook}.
In the EU as well as in the USA, approximately 11 \% of all the energy produced is used only for lighting \cite{bertoldi2009electricity} \cite{outlook2010energy}.

\begin{figure}[t]
	\centering
	\includegraphics[width=\textwidth]{chapters/introduction-chapters/energy-consumption-house.png}
	\caption{An example of energy consumption of a household over the course of a day \cite{kolter2011redd}.}
	\label{fig:energy-consumption-house}
\end{figure}




If every light in a household would have a unique power draw, that is distinguishable from all other lights, the consumer can get better insight how lighting is being used in his household. 
With this feedback, a consumer can see which individual lights are being used in the middle of the day for example or in an unoccupied room.
The consumer can than turn the lights off.
Thereby saving energy and costs.



Giving each light a unique signature is an almost intractable problem.
But the advent of smart lights can help with this.
The unique power draw of each light can be managed through VLC (Visible Light Communication).
With this information that is being sent via VLC, the light can act as a beacon for indoor localization \cite{Kuo:2014:LIP:2639108.2639109}.
For example a smart-phone could then locate itself inside a building with good accuracy.
The ID of the LED beacon which is sent via VLC will also propagate via the current that the LED draws.
The IDs must be chosen in such a way that a smart-meter is still able to identify them on a per lamp basis.
And the current that is drawn must have a shape that can be added with similar patterns so that the smart-meter can make sense of this aggregated current and can start to identify if a LED is on or off.






% !TeX root = ../../thesis.tex


\section{Problem Definition}

Energy disaggregation can detect energy consumers based off the distinct power draw that these devices have.
What it cannot yet do, is the disaggregation of appliances that have very similar power draw, such as lighting.


% !TeX root = ../../thesis.tex


\section{Thesis Contributions}

The aim of this thesis is to propose a framework consisting of theoretical methods, hardware and software, so that lights with similar power draw can be distinguished with a single smart-meter.
This thesis takes advantage of advances in two areas: CDMA codes and VLC.

The specific contributions of this thesis are:

\begin{itemize}

	\item Coding methods are analyzed to allow each LED to have a unique power draw. 
	These codes make it possible to identify if an LED is on or off, even when multiple LEDs are modulating and thereby interfering with each others unique current signature.




	\item Hardware is introduced to allow the LEDs to be modulated by a micro-controller. 
	This will allow the LEDs to propagate their unique signatures via either AC or DC. 
	And there is also hardware introduced that can sample the current at a high frequency via either AC or DC.
	The sampled current is then processed by another separate micro controller, which can then state which LEDs are on and which or off.




	\item An evaluation of the proposed hardware and software is carried out, with a testbed which uses standard LED light fixtures. For larger scale evaluations, software simulation is used. 
\end{itemize}


% maybe more ..

% !TeX root = ../../thesis.tex

\section{Thesis Organization}

The remainder of this thesis is organized as follows.
First related work is discussed in \autoref{chp:state-of-the-art}.
Then the design requirements are outlined in \autoref{chp:design-requirements}.
In \autoref{chp:cdma} the theory is explained.
Next, in \autoref{chp:hardware-design}, the design of the hardware is discussed.
In \autoref{chp:evaluation} the hardware and software is evaluated.
Finally the work is concluded in \autoref{chp:conclusionsandfuturework} and future work is identified.








