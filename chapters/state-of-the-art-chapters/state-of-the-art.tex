% !TeX root = ../../thesis.tex

\chapter{State-of-the-Art}
\label{chp:state-of-the-art}


% what is vlc and how it is used

Visible light communication (VLC) is a short-range optical means of wireless communication using the visible light spectrum.
VLC is made possible by the advances in LED technology.
The interest in VLC has grown since the widespread deployment of LED lighting fixtures for energy efficiency over the normal incandescent light bulb \cite{rajagopal2012ieee}.


Now that VLC allows lighting fixtures to transmit data, research is being done to make indoor localization more precise than with traditional RF-based approaches \cite{Kuo:2014:LIP:2639108.2639109}.
Since the light does not pass through walls and it has less reflection compared to RF, an indoor localization system can be more precise.



Many research papers have successfully recognized home appliances through energy disaggregation with data being collected with a single smart-meter.   
Though they succeed in detecting appliances with unique signatures, lighting is put in a group without mentioning the individual lights \cite{kolter2011redd}.
The reason for this, is because every light has a very similar power draw and therefor a very similar signature and it can not be reliably disaggregated.


